\documentclass{article}
\usepackage[utf8]{inputenc}
\usepackage[a4paper, top=2cm, bottom=2cm, left=2cm, right=2cm]{geometry}
\usepackage{amsfonts}
\usepackage{amsmath}
\usepackage{amssymb}
\usepackage{amsthm}
\usepackage{esint}
\usepackage{fancyhdr}
\usepackage{enumitem}
\usepackage{amsmath}
\usepackage{amsthm}
\usepackage{amssymb}
\usepackage{mathabx}
\usepackage[linesnumbered , ruled , vlined]{algorithm2e}
\usepackage{listings}
\usepackage{xcolor}
\usepackage{floatrow}
\usepackage{graphicx}
\usepackage{fancyhdr}
\usepackage{listings}
\usepackage[hypcap=false]{caption}
\usepackage{hyperref}
\usepackage{subfig}
\usepackage{tikz}
\usepackage{hyperref}

\pagestyle{fancy}
\fancyhf{}
\lhead{200050107, 200050130, 200050154, 200050157}
\rhead{CS337}
\cfoot{\thepage}

\newcommand{\B}[1]{\textbf{#1}}
\newcommand{\I}[1]{\textit{#1}}

\title{\textbf{CS337 Course Project \\ Face Recognition}}
\author{Pranjal Kushwaha, Shashwat Garg, Vedang Asgaonkar, Virendra Kabra}
\date{Autumn 2022}

\begin{document}
\begin{sloppypar}       % for overfull, etc.

    \maketitle
    \tableofcontents

    \newpage

    \section{Dataset}
    We use the Labelled Faces in the Wild (LFW) dataset. It has been taken from \href{https://www.kaggle.com/datasets/jessicali9530/lfw-dataset}{Kaggle}. There is an uneven distribution of images, with only 10 people having at least 53 images. Thus, we train the model to classify these 10 people, with our dataset containing 53 images for each. The train-test split is 80-20, and the train set is further split to create the validation set.

    \section{Model Architecture}

    We use a Convolutional Neural Network (CNN) to create a multi-class image classifier. The network is inspired from FaceNet \cite{facenet}.

    \begin{center}
        \begin{table}[!h]
            \begin{tabular}{|c|c|c|c|c|}
                \hline
                \B{Layer} & \B{In} & \B{Out} & \B{Kernel} & \B{Params}\\
                \hline \hline
                conv1 & $250\times 250\times 3$ & $123\times 123\times 64$ & $7\times 7\times 3, 2$ & 9K\\
                batchnorm1 & $123\times 123\times 64$ & $123\times 123\times 64$ & & 128\\
                relu1 & $123\times 123\times 64$ & $123\times 123\times 64$ & & 0\\
                maxpool1 & $123\times 123\times 64$ & $61\times 61\times 64$ & $2\times 2\times 64, 2$ & 0\\
                dropout1 & $61\times 61\times 64$ & $61\times 61\times 64$ & & 0\\
                \hline
                conv2 & $61\times 61\times 64$ & $61\times 61\times 128$ & $3\times 3\times 64, 1$ & 74K\\
                batchnorm2 & $61\times 61\times 128$ & $61\times 61\times 128$ & & 256\\
                relu2 & $61\times 61\times 128$ & $61\times 61\times 128$ & & 0\\
                maxpool2 & $61\times 61\times 128$ & $30\times 30\times 128$ & $2\times 2\times 128, 2$ & 0\\
                dropout2 & $30\times 30\times 128$ & $30\times 30\times 128$ & & 0\\
                \hline
                conv3 & $30\times 30\times 128$ & $30\times 30\times 256$ & $3\times 3\times 128, 1$ & 295K\\
                batchnorm3 & $30\times 30\times 256$ & $30\times 30\times 256$ & & 512\\
                relu3 & $30\times 30\times 256$ & $30\times 30\times 256$ & & 0\\
                maxpool3 & $30\times 30\times 256$ & $15\times 15\times 256$ & $2\times 2\times 128, 2$ & 0\\
                dropout3 & $15\times 15\times 256$ & $15\times 15\times 256$ & & 0\\
                \hline
                conv4 & $15\times 15\times 256$ & $15\times 15\times 64$ & $3\times 3\times 256, 1$ & 148K\\
                batchnorm4 & $15\times 15\times 64$ & $15\times 15\times 64$ & & 128\\
                relu4 & $15\times 15\times 64$ & $15\times 15\times 64$ & & 0\\
                dropout4 & $15\times 15\times 64$ & $15\times 15\times 64$ & & 0\\
                \hline
                flatten & $15\times 15\times 64$ & 14400 & & 0\\
                \hline
                fc1 & 14400 & 1024 & & 14M\\
                relu5 & 1024 & 1024 & & 0\\
                dropout5 & 1024 & 1024 & & 0\\
                \hline
                fc2 & 1024 & 64 & & 66K\\
                relu5 & 64 & 64 & & 0\\
                dropout5 & 64 & 64 & & 0\\
                \hline
                fc3 & 64 & 10 & & 650\\
                \hline \hline
                Total & & & & 15M\\
                \hline
            \end{tabular}
            \caption{\label{table-1}Model Architecture}
        \end{table}
    \end{center}

    \bibliographystyle{abbrv}
    \bibliography{biblio}

\end{sloppypar}
\end{document}